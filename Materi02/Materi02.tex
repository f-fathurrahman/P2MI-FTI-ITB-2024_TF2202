\input{PREAMBLE01}

\title{TF2202 Komputasi Rekayasa\\
Aproksimasi, Kesalahan Pemotongan dan Pembulatan}
\author{Fadjar Fathurrahman}
\date{2024}

\begin{document}

\frame{\titlepage}

\begin{frame}{Kesalahan numerik (\textit{numerical errors})}

Dalam simulasi numerik menggunakan komputer, kita tidak luput dari
kesalahan-kesalahan yang juga sering disebut sebagai
\emph{galat} (\emph{error}).

Dalam metode numerik, kita biasanya fokus pada dua jenis kesalahan:
\begin{itemize}\tightlist
\item kesalahan (galat) pemotongan (truncation error)
\item kesalahan (galat) pembulatan (round-off error)
\end{itemize}

Selain itu juga ada jenis kesalahan lain (tidak dibahas
pada kuliah ini):
\begin{itemize}
\item kesalahan logika
\item kesalahan pemrograman
\end{itemize}

\end{frame}



\begin{frame}{Definisi galat (\textit{error})}

Galat dapat didefinisikan dari hubungan berikut:
\begin{equation*}
\text{nilai benar} = \text{aproksimasi} + \text{galat}
\end{equation*}
atau:
\begin{equation*}
E_{t} = \text{nilai benar} - \text{aproksimasi}
\end{equation*}
$E_t$: galat sebenarnya (\textit{true error})

\end{frame}


\begin{frame}{Galat relatif sebenarnya (\textit{true relative error})}

Definisi galat sebenarnya tidak memperhitungkan orde atau besar
dari nilai yang sedang dibahas. Misalnya: manakah yang memiliki galat lebih
besar dari hasil berikut?
\begin{itemize}\tightlist
\item galat 1 cm dari pengukuran panjang meja
\item galat 1 cm dari pengukuran panjang jembatan atau jalan raya
\end{itemize}
Tentu saja dua galat tersebut berbeda, meskipun sama-sama bernilai
1 cm.

Alternatif yang dapat digunakan
adalah dengan \textit{galat relatif}:
\begin{equation*}
\text{galat relatif sebenarnya} = \frac{\text{galat sebenarnya}}{\text{nilai benar}}
\end{equation*}
atau dinyatakan dalam persentase
\begin{equation*}
\epsilon_{t} = \frac{\text{galat sebenarnya}}{\text{nilai benar}} \times 100\%
\end{equation*}

\end{frame}



\begin{frame}{Galat aproksimasi}

Pada kondisi riil, nilai sebenarnya biasanya tidak diketahui \textit{a priori}.
Dalam kondisi ini, kita dapat menggunakan galat aproksimasi:
\begin{equation*}
\epsilon_{a} = \frac{\text{galat aproksimasi}}{\text{aproksimasi}}
\end{equation*}

Dalam aplikasinya juga dapat digunakan definisi lain dari galat aproksimasi.
Misalnya pada kasus pendekatan \textit{iteratif} di mana aproksimasi dilakukan
berulang, nilai aproksimasi dihitung berdasarkan nilai aproksimasi sebelumnya.
Pada kasus ini kita dapat menggunakan:
\begin{equation*}
\epsilon_{a} = \frac{\text{aproksimasi sekarang} - \text{aproksimasi sebelumnya}}%
{\text{aproksimasi sekarang}} \times 100\%
\end{equation*}
Proses iteratif biasanya diulangi sampai nilai $\epsilon_{a}$ mencapai atau
lebih kecil dari nilai tertentu, misalnya $\epsilon_{s}$:
\begin{equation*}
\left| \epsilon_{a} \right| < \epsilon_{s}
\end{equation*}


\end{frame}


\begin{frame}{Kriteria Scarborough}

Scarborough mengusulkan suatu kriteria berikut:
\begin{equation*}
\epsilon_{s} = (0.5 \times 10^{2-n})\%
\end{equation*}
yang mana jika kriteria ini terpenuhi maka hasil numerik yang kita peroleh
benar sedikitnya dalam $n$ angka penting (\textit{significant figures}).

\end{frame}



\begin{frame}[fragile]{Contoh: deret Maclaurin}
\fontsize{9}{10}\selectfont

Tinjau deret Maclaurin berikut:
\begin{equation*}
e^{x} = 1 + x + \frac{x^2}{2!} + \frac{x^3}{3!} + \cdots + \frac{x^{n}}{n!}
\end{equation*}
Misalnya kita ingin menentukan nilai dari $e^{0.5}$ menggunakan deret Maclaurin ini.
Kita ingin agar hasil aproksimasi yang kita peroleh benar setidaknya 3 angka penting.
Menggunakan kriteria Scarborough kita dapat menghitung $\epsilon_s$ yang
diperlukan (dengan $n=3$):
\begin{equation*}
\epsilon_{s} = (0.5 \times 10^{2-3})\% = 0.05\% = 0.0005
\end{equation*}

Pertama, kita dapat menggunakan hanya dua suku:
$$
e^{x} = 1 + x
$$
dengan $x = 0.5$ diperoleh:
$$
e^{0.5} = 1 + 0.5 = 1.5
$$
Karena $e^{0.5}$ dapat dihitung dengan menggunakan pustaka atau fungsi \pyinline{exp}
pada Python, kita dapat menghitung galat sebenarnya dengan menganggap bahwa keluaran
dari fungsi \pyinline{exp} adalah nilai sebenarnya:
$$
\epsilon_{t} = \frac{1.648721 - 1.5}{1.648721} \times 100\% = 9.02\%
$$
\end{frame}


\begin{frame}[fragile]{Contoh: deret Maclaurin}
\fontsize{9}{10}\selectfont

Kita juga dapat menghitung galat aproksimasi:
$$
\epsilon_{a} = \frac{1.5 - 1}{1.5} \times 100\% = 33.3 \% 
$$

Nilai $\epsilon_a$ yang diperoleh masih lebih besar dari $\epsilon_s$. Kita
perlu menggunakan lebih banyak suku. Proses ini diberikan pada tabel berikut.

{\centering
\includegraphics[height=0.4\textheight]{../chapra_7th/Chapra_Table_Example_3_2.png}
\par}

Dari tabel tersebut diketahui bahwa setelah menggunakan enam suku, galat aproksimasi
menjadi lebih kecil daripada 0.05 \%.


\end{frame}



\input{galat_pembulatan}

\begin{frame}{Galat pemotongan (\textit{truncation error})}

Galat pemotongan adalah galat yang dihasilkan karena penggunaan
suatu aproksimasi dibandingkan dengan prosedur eksak.

Contohnya adalah penggunaan persamaan beda hingga untuk mengaproksimasi
turunan pertama:
$$
\frac{\partial v}{\partial t} \approx \frac{\Delta v}{\Delta t}
$$

\end{frame}



\begin{frame}{Deret Taylor}

\begin{equation}
f(x_{i+1}) = f(x_i) + f'(x_{i}) + \frac{f''(x_i)}{2!}h^2 + \cdots +
\frac{f^{(n)}(x_i)}{n!}h^n + R_{n}
\end{equation}

\begin{equation}
R_{n} = \frac{f^{(n+1)}(\xi)}{(n+1)!}h^{n+1}
\end{equation}

\end{frame}


\begin{frame}{Contoh 4.1 (Chapra)}

Gunakan ekspansi deret Taylor untuk mengaproksimasi fungsi:
$$
f(x) = -0.1 x^{4} - 0.15 x^{3} - 0.5 x^{2} - 0.25 x + 1.2
$$
di sekitar $x_{i} = 0$ dengan $h=1$ (artinya kita diminta untuk menentukan
aproksimasi dari fungsi pada saat $x_{i+1}=1$). Hitung juga galat pemotongan
yang terjadi.

\end{frame}


\begin{frame}
Secara umum, deret Taylor dengan $n$-suku akan eksak untuk polinomial
orde-$n$. Untuk fungsi lain yang diferensiabel dan kontinu, seperti eksponensial
dan sinusoid, diperlukan jumlah suku deret yang tak-hingga agar eksak.

Dalam banyak kasus, penggunaan jumlah suku yang berhingga akan menghasilkan
aproksimasi yang cukup dekat dengan nilai sebenearnya. Seberapa banyak suku
yang diperlukan agar dapat menghasilkan aproksimasi yang cukup baik bergantung
dari suku sisa $R_{n}$ dari ekspansi.

Ada beberapa hal yang perlu diingat mengenai suku sisa dari deret Taylor.
Perhatikan bahwa $\xi$ tidak diketahui, selain dari syarat bahwa $\xi$ terletak
di antara $x_{i}$ dan $x_{i+1}$. Kedua, untuk menentukan turunan ke-$n+1$ dari
$f(x)$, tentu saja kita harus mengetahui $f(x)$. Akan tetapi, jika $f(x)$ diketahui
kita tidak perlu menggunakan deret Taylor, kita cukup menggunakan $f(x)$.

Meskipun tidak dapat digunakan secara praktis, Persamaan suku sisa masih
berguna untuk memberikan gambaran mengenai galat pemotongan yang terjadi.
Hal ini karena kita dapat mengatur suku yang terkait dengan $h$ pada ekspansi
deret Taylor: kita dapat memiliki sejauh mana dari $x$ kita ingin mengevaluasi $f(x)$
dan jumlah suku yang ingin kita perhitungkan.
\end{frame}



\begin{frame}
Persamaan suku sisa biasanya dinyatakan dengan:
$$
R_{n} = \mathcal{O}(h^{n+1})
$$
yang berarti bahwa galat pemotongan yang terjadi memiliki orde $h^{n+1}$. Hal ini
berguna untuk \textit{perbandingan} dalam estimasi galat pemotongan.

Misalnya jika sisa atau galat pada deret Taylor adalah $\mathcal{O}(h)$, maka
jika ukuran langkah $h \rightarrow h/2$ maka galat $\epsilon \rightarrow \epsilon/2$. 

Pada kasus lain, jika galat deret Taylor adalah $\mathcal{O}(h^2)$ maka
jika ukuran langkah $h \rightarrow h/2$ maka galat $\epsilon \rightarrow \epsilon/4$. 
\end{frame}


\end{document}
