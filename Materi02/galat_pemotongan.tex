\begin{frame}{Galat pemotongan (\textit{truncation error})}

Galat pemotongan adalah galat yang dihasilkan karena penggunaan
suatu aproksimasi dibandingkan dengan prosedur eksak.

Contohnya adalah penggunaan persamaan beda hingga untuk mengaproksimasi
turunan pertama:
$$
\frac{\partial v}{\partial t} \approx \frac{\Delta v}{\Delta t}
$$

\end{frame}



\begin{frame}{Deret Taylor}

\begin{equation}
f(x_{i+1}) = f(x_i) + f'(x_{i}) + \frac{f''(x_i)}{2!}h^2 + \cdots +
\frac{f^{(n)}(x_i)}{n!}h^n + R_{n}
\end{equation}

\begin{equation}
R_{n} = \frac{f^{(n+1)}(\xi)}{(n+1)!}h^{n+1}
\end{equation}

\end{frame}


\begin{frame}{Contoh 4.1 (Chapra)}

Gunakan ekspansi deret Taylor untuk mengaproksimasi fungsi:
$$
f(x) = -0.1 x^{4} - 0.15 x^{3} - 0.5 x^{2} - 0.25 x + 1.2
$$
di sekitar $x_{i} = 0$ dengan $h=1$ (artinya kita diminta untuk menentukan
aproksimasi dari fungsi pada saat $x_{i+1}=1$). Hitung juga galat pemotongan
yang terjadi.

\end{frame}


\begin{frame}
Secara umum, deret Taylor dengan $n$-suku akan eksak untuk polinomial
orde-$n$. Untuk fungsi lain yang diferensiabel dan kontinu, seperti eksponensial
dan sinusoid, diperlukan jumlah suku deret yang tak-hingga agar eksak.

Dalam banyak kasus, penggunaan jumlah suku yang berhingga akan menghasilkan
aproksimasi yang cukup dekat dengan nilai sebenearnya. Seberapa banyak suku
yang diperlukan agar dapat menghasilkan aproksimasi yang cukup baik bergantung
dari suku sisa $R_{n}$ dari ekspansi.

Ada beberapa hal yang perlu diingat mengenai suku sisa dari deret Taylor.
Perhatikan bahwa $\xi$ tidak diketahui, selain dari syarat bahwa $\xi$ terletak
di antara $x_{i}$ dan $x_{i+1}$. Kedua, untuk menentukan turunan ke-$n+1$ dari
$f(x)$, tentu saja kita harus mengetahui $f(x)$. Akan tetapi, jika $f(x)$ diketahui
kita tidak perlu menggunakan deret Taylor, kita cukup menggunakan $f(x)$.

Meskipun tidak dapat digunakan secara praktis, Persamaan suku sisa masih
berguna untuk memberikan gambaran mengenai galat pemotongan yang terjadi.
Hal ini karena kita dapat mengatur suku yang terkait dengan $h$ pada ekspansi
deret Taylor: kita dapat memiliki sejauh mana dari $x$ kita ingin mengevaluasi $f(x)$
dan jumlah suku yang ingin kita perhitungkan.
\end{frame}



\begin{frame}
Persamaan suku sisa biasanya dinyatakan dengan:
$$
R_{n} = \mathcal{O}(h^{n+1})
$$
yang berarti bahwa galat pemotongan yang terjadi memiliki orde $h^{n+1}$. Hal ini
berguna untuk \textit{perbandingan} dalam estimasi galat pemotongan.

Misalnya jika sisa atau galat pada deret Taylor adalah $\mathcal{O}(h)$, maka
jika ukuran langkah $h \rightarrow h/2$ maka galat $\epsilon \rightarrow \epsilon/2$. 

Pada kasus lain, jika galat deret Taylor adalah $\mathcal{O}(h^2)$ maka
jika ukuran langkah $h \rightarrow h/2$ maka galat $\epsilon \rightarrow \epsilon/4$. 
\end{frame}