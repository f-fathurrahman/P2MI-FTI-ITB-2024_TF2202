\begin{frame}{Metode iterasi titik-tetap (\textit{fixed-point} iteration)}
\fontsize{10}{11}\selectfont

Metode iterasi titik-tetap adalah salah satu dari metode terbuka untuk menghitung
akar dari persamaan nonlinear.
Langkah pertama dari metode ini adalah menuliskan kembali persamaan
$f(x) = 0$ menjadi:
\begin{equation}
x = g(x)  
\label{eq:x_gx}
\end{equation}
Jika $x$ merupakan akar dari $f(x)$ maka Persamaan \eqref{eq:x_gx}
juga akan terpenuhi. Artinya jika kita dapat menemukan $x$ sedemikan rupa sehingga
$x = g(x)$, maka kita telah menemukan akar dari $f(x)$.
Dalam bentuk iterasi, metode ini dapat dituliskan sebagai:
$$
x_{i+1} = g(x_{i})
$$
di mana $i$ adalah indeks iterasi. Terdapat beberapa kondisi yang dapat
digunakan untuk menentukan konvergensi. Salah satunya adalah dengan
menggunakan estimasi galat
$$
\epsilon_{a} = \left| \frac{x_{i+1} - x_{i}}{x_{i+1}} \right|
$$
Iterasi dihentikan jika $\epsilon_{a}$ lebih kecil dari suatu nilai
tertentu.
\end{frame}


