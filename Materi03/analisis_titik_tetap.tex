\begin{frame}
\fontsize{9}{10}\selectfont

Skema iterasi:
$$
x_{i+1} = g(x_{i})
$$
Misalkan solusi dari iterasi titik-tetap ini adalah $x_r$, sehingga berlaku:
$$
x_{r} = g(x_r)
$$
Dengan mengurangi kedua persamaan tersebut diperoleh:
\begin{equation}
x_r - x_{i+1} = g(x_r) - g(x_i)
\label{eq:B6_1_1}
\end{equation}

Menggunakan teorema nilai rata-rata turunan: jika suatu fungsi $g(x)$
dan turunan pertamanya kontinu pada suatu interval $a \leq x \leq b$,
maka terdapat setidaknya satu nilai $x = \xi$ dalam interval tersebut
sedemikian rupa sehingga:
\begin{equation}
g'(\xi) = \frac{g(b) - g(a)}{b - a}
\label{eq:B6_1_2}  
\end{equation}
Ruas kanan dari persamaan ini adalah kemiringan dari garis yang menghubungkan antara
$g(a)$ dan $g(b)$. Dengan kata lain, teorema rata-rata turunan menyatakan bahwa
setidaknya ada satu titik di antara $a$ dan $b$ yang memiliki kemiringan,
$g'(\xi)$, yang sejajar dengan garis yang menghubugkan antara $g(a)$ dan
$g(b)$.
\end{frame}


\begin{frame}
\fontsize{9}{10}\selectfont

Jika $a = x_{i}$ dan $b = x_r$, maka ruas kanan dari
Persamaan \eqref{eq:B6_1_1} dapat dituliskan sebagai:
\begin{equation*}
g(x_r) - g(x_i) = (x_r - x_i) g'(\xi)
\end{equation*}
sehingga Persamaan \eqref{eq:B6_1_1} dapat dituliskan juga menjadi:
\begin{equation}
x_r - x_{i+1} = (x_r - x_i) g'(\xi)
\end{equation}
Dengan definisi galat sebenarnya untuk iterasi ke-$i$ sebagai:
$$
E_{t,i} = x_r - x_i
$$
maka:
$$
E_{t,i+1} = g'(\xi) E_{t,i}
$$
Dari persamaan ini dapat dilihat bahwa galat akan semakin berkurang pada
setiap iterasi jika $|g'(x)| < 1$. Sebaliknya, jika $|g'(x)| > 1$
galat akan semakin membesar.

Selain itu, jika turunan bernilai positif, maka galat akan positif solusi iteratif
akan monotonik. Jika turunan bernilai negatif, maka galat akan berosilasi.

Analisis ini juga menunjukkan bahwa galat pada metode iterasi titik-tetap akan
sebanding dengan galat dari langkah sebelumnya. Oleh karena ini iterasi titik-tetap
dikatakan memiliki sifat konvergensi linear (\textit{linearly convergent}).
\end{frame}