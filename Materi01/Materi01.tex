\input{PREAMBLE01}

\title{TF2202 Komputasi Rekayasa\\
Pendahuluan}
\author{Fadjar Fathurrahman}
\date{2024}

\begin{document}

\frame{\titlepage}

\begin{frame}{Sekilas mengenai kuliah ini}

\begin{itemize}
\item Kode dan nama kuliah: Komputasi Rekayasa
\item Nama lain: metode numerik, komputasi teknik
\item Dalam kuliah ini kita akan mempelajari \emph{metode-metode numerik}
  yang diimplementasikan dengan menggunakan komputer dengan aplikasi
  yang sesuai dan/atau bahasa pemrograman. Metode-metode numerik ini
  akan digunakan untuk menyelesaikan \emph{persamaan matematis} yang
  muncul pada pemodelan masalah yang muncul pada sains dan keteknikan.
\end{itemize}
\end{frame}




\begin{frame}{Sekilas mengenai kuliah ini}

\begin{itemize}
\item Beban: 3 SKS
\item Referensi utama: Chapra and Canale. Numerical Methods for Engineers. (7th Ed.)
\item Praktikum (berkelompok, jadwal akan ditentukan dengan asisten)
\item Kuliah ini sangat menekankan pada metode dan penguasaan \textit{tools}
berbasis komputer: dapat berupa aplikasi maupun bahasa pemrograman.
Ada beberapa tools yang dapat kita gunakan: Python, Julia, MATLAB, Fortran, dan
\emph{spreadsheet}
\item Disarankan menyiapkan laptop pada saat kuliah (opsional)
\end{itemize}

\end{frame}


\begin{frame}{Mengapa mempelajari metode numerik?}

Metode numerik diperlukan pada saat solusi analitik tidak ada
atau sulit untuk diperoleh.

Metode numerik banyak dipakai di berbagai bidang: sains (fisika, kimia, biologi),
keteknikan, ekonomi, dan bahkan sains sosial. Metode numerik dipakai
hampir di seluruh bidang yang memerlukan analisis kuantitatif.

Dalam karir yang Anda tekuni nanti, mungkin Anda perlu menggunakan program
komputer khusus yang melibatkan perhitungan numerik. Penggunaan yang efektif
dari program tersebut biasanya memerlukan pengetahuan dan pengalaman
yang terkait metode numerik, selain dari pengetahuan mengenai bidang atau masalah
khusus yang sedang ditekuni.

\end{frame}


\begin{frame}{Mengapa mempelajari metode numerik}

Ada banyak permasalahan yang belum dapat diselesaikan dengan menggunakan
program komputer yang sudah ada. Pada kondisi ini, kita perlu membuat program
kita sendiri (\textit{customized program}).
Pengalaman dalam mengimplementasikan suatu metode numerik
atau algoritma akan sangat membantu dalam kasus ini.

Implementasi metode numerik merupakan salah satu cara yang efektif untuk
belajar menggunakan komputer atau pemrograman.


\end{frame}

% Jacob VanWagoner
% https://www.quora.com/What-do-theoretical-and-experimental-physicists-think-of-each-other
% My answer comes as a joke told to me by my co-worker, who is a physicist.
% A theoretical physicist goes to a conference and presents a paper.
%    Everyone thinks he is full of crap... except himself.
% An experimental physicist goes to a conference and presents a paper.
%    Everyone thinks he is right... except himself.
% A computational physicist goes to a conference and presents a paper.
%    Everyone thinks he is full of crap... including himself.


\begin{frame}{Pemrograman}

\begin{itemize}
\item Python adalah bahasa pemrograman utama yang akan kita gunakan
\item Dapat menggunakan Jupyter Notebook; Google Colab juga dapat digunakan
\item \textit{But I cannot do (or hate) programming}:
Anda dapat menggunakan \textit{Google search} dan \textit{ChatGPT}
\end{itemize}

\end{frame}



\begin{frame}{Langkah-langkah penyelesaian masalah}

Kuliah ini sangat berkaitan erat dengan \textit{pemodelan fisis}
dan \textit{simulasi}.

\begin{enumerate}[(1)]
\item Identifikasi permasalahan
\item Pemodelan: hukum fisika, asumsi dan aproksimasi fisis penurunan
  persamaan matematika
\item Aproksimasi numerik, penurunan persamaan diskrit
\item Implementasi pada komputer (menggunakan perangkat lunak khusus atau
  dengan menulis program) untuk mendapatkan solusi
\item Visualisasi solusi
\item Analisis (termasuk validasi dan verifikasi)
\end{enumerate}

Pada kuliah ini kita akan lebih fokus pada Langkah (3), (4) dan (5)
\end{frame}


% Briefly, verification is the assessment of the software correctness and
% numerical accuracy of the solution to a given computational model.
% Validation is the assessment of the physical accuracy of a computational
% model based on comparisons between computational simulations and experimental data.


\begin{frame}{Contoh: \emph{bungee jumper}}

\fontsize{9pt}{10pt}\selectfont

\begin{columns}

\begin{column}{0.2\textwidth}

\includegraphics[height=0.7\textheight]{../chapra_python/Chapra_Fig_1_1.png}

\end{column}

\begin{column}{0.8\textwidth}
Misalkan kita diminta untuk menghitung kecepatan $v(t)$ dari pelompat
\emph{bungee} yang sedang terjun. Untuk mendapatkan ini, kita
menggunakan Hukum Newton:
\begin{equation*}
F = m \frac{\mathrm{d}}{\mathrm{d}t} v(t)
\end{equation*}
dengan $F$ adalah gaya total dan $m$ adalah massa pelompat.
Gaya yang bekerja pada penerjun ada dua, yaitu gaya gravitasi \(F_{g}\) dan
gaya akibat gesekan udara (\emph{drag})
$F_{d}$:
\begin{equation*}
F = F_{g} + F_{d}
\end{equation*}
dengan: $F_{g} = mg$, dengan $g$ adalah percepatan gravitasi dan gaya
gesekan diudara diasumsikan berbanding lurus dengan kecepatan:
\begin{equation*}
F_{d} = -c v  
\end{equation*}
dengan $c$ adalah suatu konstanta.
\end{column}

\end{columns}

\end{frame}


\begin{frame}{Contoh: \emph{bungee jumper}}

Sehingga dapat diperoleh persamaan gerak untuk \emph{bungee jumper}:
\begin{equation}
\frac{\mathrm{d}v}{\mathrm{d}t} = g - \frac{c}{m}v
\label{eq:diff-eq-01}
\end{equation}

Bagaimana cara menyelesaikan persamaan diferensial ini? Kita dapat
menggunakan kalkulus untuk mencari solusi analitiknya. Dengan
menggunakan informasi tambahan (syarat awal) bahwa $v(t=0)=0$, dapat
diperoleh solusi analitik:
\begin{equation}
v(t) = \frac{gm}{c}\left( 1 - e^{-(c/m)t} \right)
\label{eq:sol-diff-eq-01}
\end{equation}
Untuk verifikasi bahwa
persamaan ini merupakan solusi substitusi
Pers. \eqref{eq:sol-diff-eq-01} ke Pers. \eqref{eq:diff-eq-01}
\end{frame}



\begin{frame}{Solusi analitik}

\fontsize{9pt}{10pt}\selectfont

Dengan menggunakan nilai-nilai berikut: $m = 68.1$ kg, $g = 9.81$
$\mathrm{m}/\mathrm{s}^2$, $c = 12.5$ kg/s
kita dapat menghitung kecepatan sebagai fungsi dari waktu $t$.
Contoh dapat dilihat pada tabel berikut.

{\centering
\includegraphics[height=0.5\textheight]{../chapra_7th/Chapra_Example_1_1_table.png}
\par}

Kecepatan pada $t \rightarrow \infty$ dikenal juga sebagai kecepatan akhir
(\emph{terminal velocity}).

TUGAS: Tulis program Python untuk membuat plot dari $v(t)$ terhadap $t$.

\end{frame}



\begin{frame}{Solusi numerik}
Solusi yang baru saja kita dapatkan adalah solusi eksak atau solusi
analitik.

Bagaimana jika:

\begin{itemize}\tightlist
\item kita lupa cara mengintegralkan atau menyelesaikan persamaan
  diferensial yang diperoleh?
\item persamaan diferensial yang diperoleh rumit sehingga tidak ada
  solusi analitik yang tersedia
\end{itemize}

Alternatif yang dapat kita gunakan adalah dengan menggunakan
\emph{metode numerik}.
\end{frame}



\begin{frame}{Aproksimasi dan Metode numerik}

\begin{columns}[T]
\begin{column}{0.5\textwidth}
{\centering
\includegraphics[width=1\textwidth,height=\textheight]{../chapra_python/Chapra_Fig_1_3.png}
\par}
\end{column}

\begin{column}{0.5\textwidth}
Dengan menggunakan metode numerik, kita dapat mengaproksimasi turunan
dengan menggunakan:
\begin{equation*}
\frac{\mathrm{d}v}{\mathrm{d}t} \approx \frac{v_{i+1} - v_{i}}{t_{i+1} - t_{i}}
\end{equation*}

Perhatikan bahwa kita telah \emph{mendiskritisasi} kecepatan $v(t)$
dan waktu $t$ pada persamaan di atas, artinya $v$ dan $t$ tidak
lagi memiliki nilai yang kontinu, namun diskrit, seperti pada nilai
dalam tabel. Solusi numerik yang akan kita dapatkan hanya tersedia pada
titik-titik diskrit tersebut.
\end{column}
\end{columns}
\end{frame}




\begin{frame}{Solusi numerik}

\fontsize{9pt}{10pt}\selectfont

Pers. \eqref{eq:diff-eq-01} dapat
diaproksimasi menjadi:
\begin{equation*}
\frac{v_{i+1} - v_{i}}{t_{i+1} - t_{i}} = g - \frac{c}{m} v_{i}
\end{equation*}
atau:
\begin{equation*}
v_{i+1} = v_{i} + \underbrace{\left( g - \frac{c}{m} v_{i} \right)}_{
  \left.\dfrac{\mathrm{d}v}{\mathrm{d}t}\right|_{t_{i},v_{i}}
}
\underbrace{\left( t_{i+1} - t_{i} \right)}_{\Delta t}
\end{equation*}

Metode ini juga dapat dinyatakan dengan deskripsi berikut:
\begin{equation*}
\text{nilai baru} = \text{nilai lama} + \text{kemiringan}\times\text{ukuran langkah}
\end{equation*}
Metode ini dikenal dengan \emph{metode Euler}. Kita akan mempelajari
lebih lanjut mengenai metode ini pada bab mengenai persamaan diferensial
(persoalan nilai awal).
\end{frame}



\begin{frame}{Solusi numerik}

\fontsize{9pt}{10pt}\selectfont

\begin{columns}[T]
\begin{column}{0.5\textwidth}
Dengan menggunakan ukuran langkah $\Delta t = 2$ s, $t_0 = 0$,
\(v_0 = 0\) kita dapat menghitung:
\begin{eqnarray*}
v_1 = 0 + \left[ 9.81 - \frac{12.5}{68.1}(0) \right] 2 = 19.62 \\
v_2 = 19.62 + \left[ 9.81 - \frac{12.5}{68.1}(19.62) \right] 2 = 32.04
\end{eqnarray*}

Lanjutkan ...

Tipe perhitungan \emph{berulang} (dan membosankan)
ini akan sering kita temui pada kuliah ini.

Perhitungan berulang ini adalah salah satu alasan kita
menggunakan komputer.


\end{column}

\begin{column}{0.5\textwidth}

Dengan meneruskan perhitungan ini dapat diperoleh tabel berikut.

{\centering
\includegraphics[height=0.4\textheight]{../chapra_7th/Chapra_Example_1_2_table.png}
\par}

Pertanyaan: bagaimana jika kita ingin mengetahui nilai $v$ pada waktu yang
tidak tertulis pada tabel, misalnya $t=3$?

\end{column}

\end{columns}

\end{frame}



\begin{frame}{Solusi analitik vs solusi numerik}

\begin{columns}[T]

\begin{column}{0.5\textwidth}
{\centering
\includegraphics[height=0.8\textheight]{../chapra_7th/Chapra_Fig_1_5.png}
\par}
\end{column}

\begin{column}{0.5\textwidth}
Solusi numerik tidak sama persis dengan solusi analitik namun memiliki
karakteristik yang mirip.
\end{column}

\end{columns}

\end{frame}



\begin{frame}{Apa yang akan dipelajari}

\begin{itemize}
\item Aproksimasi, galat pemotongan dan pembulatan
\item Pencarian akar persamaan nonlinear
\item Solusi sistem persamaan linear
\item Interpolasi dan regresi (pencocokan kurva)
\item Optimisasi
\item Integrasi dan diferensiasi numerik
\item Solusi numerik dari persamaan diferensial biasa (masalah nilai
awal dan masalah nilai batas)
\item Solusi numerik dari persamaan diferensial parsial
\end{itemize}

\end{frame}



\begin{frame}{Penggunaan pustaka atau menulis solusi sendiri}

\textit{Use prepackaged library or roll out your own solution?}

Pada kelas ini kita akan menggunakan kedua pendekatan tersebut.

Untuk metode yang relatif sederhana biasanya Anda akan diminta
untuk mengimplementasikan sendiri program (tanpa bantuan pustaka).

Beberapa metode numerik cukup rumit sehingga tidak praktis atau
tidak mudah untuk diprogram.
Beberapa operasi umum
seperti perkalian matriks, meskipun tampak sederhana, memerlukan
keahlian khusus agar program yang dihasilkan memiliki performa
yang optimal. Untuk kasus-kasus tersebut, kita akan menggunakan
\textit{prepackaged library} atau \textit{built-in solution} yang
tersedia pada bahasa pemrograman yang digunakan.

\end{frame}



\begin{frame}{Beberapa pustaka Python yang akan digunakan}

\begin{itemize}
\item Numpy
\item Matplotlib
\item SciPy
\item SymPy
\end{itemize}

\end{frame}


\begin{frame}

Beberapa metode numerik memiliki dasar atau penurunan yang sederhana,
namun banyak metode numerik yang tidak memiliki penurunan yang pasti.
  
Sebagian besar pengguna dapat menganggap metode
numerik sebagai suatu resep atau \textit{black box} yang tinggal
diaplikasikan.
  
Intuisi dan pengalaman juga berpengaruh pada kesuksesan pada implementasi
dan penerapan metode numerik

Hasil dari suatu metode numerik harus selalu diuji: variasi parameter
pada algoritma, perbandingan dengan
metode numerik lain, solusi analitik, prinsip fisis, eksperimen dan data, atau
dengan cara lain.
  
\end{frame}



\begin{frame}{Latihan}

\fontsize{9pt}{10pt}\selectfont

Ulangi kasus untuk peloncat \emph{bungee} pada contoh sebelumnya, jika
sekarang gaya gesekan yang bekerja berbanding lurus dengan \emph{kuadrat
kecepatan}
\begin{equation*}
F_{d} = c v^2
\end{equation*}
dengan $c$ adalah konstanta atau koefisien gesekan.
Gunakan $c=2.5$ (apakah satuan dari $c$?) dan nilai-nilai yang sama dengan
contoh untuk nilai-nilai yang lain.

Bandingkan solusi numerik
yang Anda dapatkan dengan solusi analitik:
\begin{equation*}
v(t) = \sqrt{\frac{g m}{c}}\tanh\left( \sqrt{\frac{g c}{m}} t \right)
\end{equation*}
dengan $\tanh(x)$ adalah fungsi tangen hiperbolik:
\begin{equation*}
\tanh(x) = \frac{e^x - e^{-x}}{e^{x} + e^{-x}}
\end{equation*}

\end{frame}



\end{document}
